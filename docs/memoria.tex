\documentclass[12pt]{article}
\usepackage[utf8]{inputenc}
\usepackage[spanish]{babel}
\usepackage{graphicx}
\usepackage[hidelinks]{hyperref}
\usepackage[left=3cm, right=3cm, top=2cm]{geometry}

\newcommand{\HRule}{\rule{\linewidth}{0.5mm}}
\title{\textbf{\huge Memoria - AA}}
\author{{David Rodríguez Bacelar} \\[0.25cm] {Kevin Millán Canchapoma} \\[0.25cm]{Luca D'angelo Sabín} \\[0.25cm]{Jorge Hermo González}}
\date{\today}

\begin{document}

\maketitle
\HRule
\bigskip\bigskip\bigskip\bigskip\bigskip\bigskip
\begin{figure}[h!]
	\centering
	\includegraphics[height=180px]{udc.jpg}
	\label{fig:diagram1}
\end{figure}

\newpage

\tableofcontents

\newpage
\section{Introducción}

A raíz de la pandemia, el aumento del interés por el aprendizaje en diferentes ámbitos llegó tambíen a la música y con él, la aparición
de herramientas para aprender a tocar diferentes instrumentos de forma autodidacta.

\bigskip
Así, para cualquiera que esté aprendiendo, el escuchar una canción que te gusta e intentar tocarla es algo que acaba siendo un proceso
frustante y que requiere una gran cantidad de horas intentando sacar las notas que la componen.

\bigskip
Nuestro sistema se encargaría entonces de reconocer y diferenciar a partir de audios, las notas de una pieza de piano
pudiendo, en un futuro, ser capaz de detectar acordes, tonalidades e incluso generar la propia partitura, siendo útil en aplicaciones como Spotify, Tidal...

\bigskip
A lo largo de esta memoria analizaremos a fondo el problema a resolver en la Sección \ref{Descripción del problema}, desarrollaremos las diferentes soluciones en la
Sección \ref{Desarrollo}, hablaremos sobre las conclusiones del trabajo en la Sección \ref{Conclusiones} y finalizaremos comentando las aplicaciones al mundo
real en la Sección \ref{Trabajo futuro}. También se podrá consultar las bibliografía utilizada en la Sección \ref{Análisis bibliográfico} y \ref{Bibliografía}.

\newpage

\section{Descripción del problema}
\label{Descripción del problema}

Nuestro sistema se centrará en reconocer, a partir de un audio, la nota del piano que se está tocando. Escogimos este instrumento por la cantidad de recursos
que podemos encontrar y por su naturaleza invariable al ser tocada por una u otra persona.

\bigskip
Como única restricción, en dicho audio solo puede haber una nota sonando a la vez para que el sistema sea capaz de reconocerla correctamente.

\bigskip
La base de datos con la que contamos tiene un total de 5.417 audios con más de 50 samples por cada una de las 88 nota del piano, tocadas
desde posiciones e intensidades distintas y grabadas con micrófonos diferentes. Dichos samples están en formato \textit{.wav} en estéreo con un 
bitrate de \textit{2304kbps}, \textit{24 bits per sample} y un sample rate de \textit{48kHz}. Todo ello ocupa un total de 34.5GB en disco.

\bigskip
El origen de la base de datos es una librería de piano de la compañía FluffyAudio \textit{\url{https://www.fluffyaudio.com/shop/scoringpiano/}} 
grabada en 2016 y pensada para jazz, música clásica y bandas sonoras.

\section{Análisis bibliográfico}
\label{Análisis bibliográfico}

Trabajos como el de J Osmalskyj, J-J Embrechts, S Piérard, M van Droogenbroeck (2012). \textit{Neural Networks for Musical Chords Recognition}, ofrecen otro enfoque
en el que, en lugar de detectar las notas por separado, analizan todo el espectro de frecuencias para poder reconocer acordes completos de diferentes instrumentos.
Para ello, utilizan una técnica denominada Pitch Class Profile (PCP) que obtiene las relaciones energéticas de cada nota en la escala.

\bigskip
Además, como resumen Emmanouil Benetos Member, Simon Dixon, Zhiyao Duan Member, Sebastian Ewert Member (2019). \textit{Automatic Music Transcription: An Overview},
a pasar del estado avanzado de la transcripción automática de canciones, aún están presentes retos tales como la independencia de los intrumentos, de los estilos
musicales o la interpretación de la expresividad.

\bigskip
Otros trabajos mas antiguos como los de Foo S. W., Wong P. L. (1999). \textit{Recogition of piano notes}, describen un algoritmo capaz de reconocer notas
de un piano a partir de piezas sintetizadas o acústicas que son digitalmente muestreadas y transformadas al dominio de frecuencia usando 
la transformada de Q constante a partir de la cual se la aplican diferentes técnicas para identificar las notas.

\section{Desarrollo}
\label{Desarrollo}

\subsection{Primera aproximación}
\label{Primera aproximación}

\section{Conclusiones}
\label{Conclusiones}

\section{Trabajo futuro}
\label{Trabajo futuro}

\section{Bibliografía}
\label{Bibliografía}


\end{document}
