\documentclass{article}
\usepackage[utf8]{inputenc}
\usepackage[spanish]{babel}
\usepackage{graphicx}
\usepackage[left=3cm, right=3cm, top=2cm]{geometry}

\newcommand{\HRule}{\rule{\linewidth}{0.5mm}}
\title{\textbf{\huge Memoria - AA}}
\author{{David Rodríguez Bacelar} \\[0.25cm] {Kevin Millán Canchapoma} \\[0.25cm]{Luca D'angelo Sabín} \\[0.25cm]{Jorge Hermo González}}
\date{\today}

\begin{document}

\maketitle
\HRule
\bigskip\bigskip\bigskip\bigskip\bigskip\bigskip
\begin{figure}[h!]
	\centering
	\includegraphics[height=180px]{udc.jpg}
	\label{fig:diagram1}
\end{figure}

\newpage

\tableofcontents

\newpage
\section{Introducción}

\subsection{Descripción del problema}
A raíz de la pandemia, el aumento del interés por el aprendizaje en diferentes ámbitos llegó tambíen a la música y con él, la aparición
de herramientas didácticas y profesionales para aprender a tocar diferentes instrumentos.

Así, para cualquiera que esté aprendiendo, el escuchar una canción que te gusta e intentar tocarla es algo que acaba siendo un proceso
frustante y que requiere una gran cantidad de horas intentando sacar las notas que la componen.

Nuestro sistema se encargaría entonces de reconocer y diferenciar a partir de audios, las notas de una pieza de piano
pudiendo, en un futuro, ser capaz de detectar acordes, tonalidades e incluso generar la propia partitura pudiéndose 
utilizar en aplicaciones como Spotify, Tidal...


\section{Descripción del problema}

\section{Análisis bibliográfico}

\section{Desarrollo}

\section{Conclusiones}

\section{Trabajo futuro}

\section{Bibliografía}



\end{document}
